\documentclass{article}

\usepackage{listings}

\title{A tactic for normalising ring expressions with exponents}

\newcommand{\ringexp}{\texttt{ring\_exp}}

\begin{document}
\maketitle

Abstract:
This paper describes the design of the normalising tactic \ringexp for the Lean prover.
This tactic improves on existing tactics by adding a binary exponent operator to the language of rings.
A normal form is represented with an inductive family of types, enforcing various invariants.
The design can also be extended with more operators.

Introduction
------------

We can use a normalising tactic to prove equalities.
Examples in mathlib: `norm\_cast`, `norm\_num`, `ring`, `simp`, ...

The Lean mathematical library already had a tactic `ring` for commutative rings.
The `ring` tactic is useful: occurs $N$ times in the mathlib code base.
Can also be used as components of other tactics: `linarith`, `omega`, ...

Horner normal forms cannot represent compound exponents.

Exponents also result in heterogeneous expressions: $a^b$ has $a : α$ for any ring, but $b : ℕ$.

The goal was to make a tactic \ringexp whose domain is a strict superset of `ring`'s.
This was possible without sacrificing the efficiency of `ring`.
An additional result is that the design is extensible to more operators.

Overview of the tactic
----------------------

The \ringexp tactic uses a similar normalisation scheme as the original `ring` tactic.
The input to the normaliser is an abstract syntax tree representation of the expression to normalise.
% Lean makes the syntax available for metaprogramming as terms of type `expr`.
The input is evaluated by the `eval` function to give a term in an inductive type of normalised expressions, here called `ex`.
From the `ex` representation, the normalised output expression is constructed by the `simple` function,
which also returns a proof that the in- and output expression are equal.
The normal form should be designed in such a way that values of type `ex` are equal if and only if the input expressions are equal.
%The `ex` representation is also used by an additional tactic `ring_exp_eq` to test expressions for equality.

In addition to the language of semirings implemented by `ring`, with binary operators $+$, $*$ and optionally $-$ and $/$,
the \ringexp tactic supports a binary exponentiation operator $\^$.
The input expression can consist of any of these operators applied to other expressions,
with two base cases: rational numerals such as `0`, `37` and `2/3` and atoms.
An atom is any expression which is not of the above form, e.g. a variable name `x` or a function application `f (x - 2)`,
which are treated as arbitrary constants in the expression.
The language parsed by \ringexp should not be confused with that of an `exponential ring', which is a ring $(R, +, *)$ equipped with a unary operator $E$ which is a monoid homomorphism $(R, +) \to (R^*, *)$.

%For instance, each expression in the language of monoids (with associative operator $*$ and neutral element $1$) has a list of atoms as normal form;
%the conversion interprets $*$ as list concatenation `++` and $1$ as the empty list `nil`.
%Adding an associative, distributive $+$ operator will give the language of semirings,
%and by distributivity, we can always rewrite in such a way that the arguments to $*$ are not sums.
%This allows us to write a polynomial as a sum of monomials, which are themselves products of atoms.
%The exponentation operator, however, does not have such nice associativity or distributivity properties:
%in general, only `(a ^ b)^c = a ^ (b * c)` and `(a * b)^c = a^c * b^c` hold.

Using a suitable representation of the normal form is crucial for correctness of the normaliser.
The `ex` normal form is a tree with operators at the nodes and atoms at the leaves,
with certain restrictions on which subnodes may occur for each node.
Compared to the abstract syntax tree, this will prohibit subexpressions that are not normalised.
For example, associativity allows rewriting $(a * b) * c$ to $a * (b * c)$ and distributivity allows rewriting $(a + b) * c$ to $(a * c) + (b * c)$,
which leads to the rule that the left argument to $*$ cannot be of the form $a + b$ or $a * b$.
These restrictions are expressed in the `ex` type by parametrising it over the enum `ex\_type`,
giving an inductive family of types.
Each constructor only allows certain members of the `ex` family in its arguments,
and returns a specific type of `ex`:

\begin{lstlisting}
	meta inductive ex : ex_type → Type
	| zero  :                     ex sum
	| sum   : ex prod → ex sum  → ex sum
	| coeff : coeff             → ex prod
	| prod  : ex exp  → ex prod → ex prod
	| exp   : ex base → ex prod → ex exp
	| var   : atom              → ex base
	| sum_b : ex sum            → ex base
\end{lstlisting}

Here, the `sum` constructor represents $+$, `prod` represents $*$ and `exp` represents $\^$,
%the `zero` and `coeff` constructors represent zero and non-zero numeric coefficients,
the `var` constructor represents an atom and `sum\_b` allows sums as the base of expressions, analogous to the brackets in $(a + b) ^ c$.
Thus, the restriction that $+$ and $*$ cannot occur as the left argument to $*$,
is reflected by the first argument to `prod` being `ex exp`, which must be of the form $a ^ b$.
Similarly, the `sum` constructor has arguments `ex prod` and `ex sum`, reflecting that the $+$ operator is re-associated to the right in the normal form.
% Some further examples on how the equalities are reflected in the types?

% TODO: explain this better:
Not only does `ex` prohibit a `sum` as the left argument to `sum`,
it also prohibits `exp`.
The reason for this can be read from the matrix of equalities.
Putting the outermost operator in a matrix gives:
\begin{tabular}{l l l l}
$l \backslash r$	& $+$	& $*$	& $\^$	\\
$+$	& $+$ $+$	& $+$ $+$	& $+$ $\^$	\\
$*$	& $+$ $+$	& $*$ $*$	& $*$ $*$	\\
$\^$	& $*$ $+$	& $\^$ $*$	& $\^$ $\^$	\\
\end{tabular}
From the matrix, we can see that sums cannot be rewritten as a product or exponent
and products cannot be rewritten as exponents,
but the other way can occur: the distributivity rule $a * (b + c) = (a * b) + (a * c)$ rewrites a product into a sum.
Thus, allowing for exponentiation implies allowing for multiplication, which implies allowing for addition.

While the `ex` type enforces that some normalisation rules are always applied,
others cannot be easily expressed in the type system.
For instance, the $+$ and $*$ operators are also commutative,
but this is not reflected in the definition of the `ex` type.
In Lean, testing for (definitional) equality of expressions is a monadic operation,
so it is impossible to take the quotient modulo different permutations of the list of terms.
Instead, the operations on `ex` have to manually maintain the invariants that guarantee unique representation.

Some examples where the types are incomplete, and how the tactic mitigates that.

The `ex\_info` type allows us to (re)construct the proof that normalisation is correct.
For each operation, instantiate correctness proof (properties are subtle to express) and combine these while evaluating.

The setup makes it easy to extend `ex` with another operator by adding a new constructor.

Complications
-------------

Dealing with atoms (expressions not parsed by \ringexp):
keep a list of all unique (modulo definitional equality) atoms encountered, compare on index into list (not structure of atom!).

Commutativity: commutative operators sort arguments by picking an order on `ex`.
Order should be consistent (Lean's expr.has\_lt is not): order atoms by appearance, the rest lexicographically.

Dealing with $a ^ b$: interpret $b$ as a natural number (integer?):
use a reader monad to keep track of the expression's inferred type.

Dealing with numerals: don't unfold `2 * a` to `a + a`, but fold `1 * a + 1 * a` to `2 * a`.
But when adding, the tactic needs to keep track of overlap.

Dealing with $a - b$: rewrite to $a + (-1) * b$.
Dealing with $a / b$: rewrite to $a * b^(-1)$.

Normalising atoms as well: \ringexp feeds the normaliser to `simp`.

Optimisations
-------------

An important reason to prefer \ringexp over `simp` is that \ringexp should be faster.
It will be called often as part of various other tactics: `linarith`, ...

The `ring` tactic is successful, so it will be the reference point.

Optimisations:
Making the tactic fast enough took some work:
- The tactic precomputes typeclass instances because typeclass inference can be very expensive.
- The numerals `0` and `1` are also precomputed.
- Although the proofs can take implicit arguments, they are all made explicit.
- During construction of an `ex`, the correctness proof ``migrates outward'', saving memory by discarding the proofs in subexpressions.

Results of optimising:
Empirically, there is only a constant factor slowdown compared to `ring`:
1.6 times in the case of linear expressions (i.e. `linarith`).
Faster than `ring` for e.g. $(x + 1)^4 = (1 + x)^4$.

Some graphs with nice regression lines.

To test, replace the implementation of `ring` with \ringexp in mathlib: no errors, same compilation time.

Discussion
----------

The \ringexp tactic can deal with a strict superset of expressions,
and does so without sacrificing too much speed.

The approach should be adaptable to extra operators (gcd? min/max?) and similar problem domains (lattices? propositions?).
Can we automate the construction of the `ex` type? (Probably not?)
\end{document}
